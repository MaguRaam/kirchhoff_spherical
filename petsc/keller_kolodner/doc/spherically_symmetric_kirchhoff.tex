\documentclass[a4paper]{article}

\usepackage[english]{babel}
\usepackage[utf8]{inputenc}
\usepackage{csquotes}
\usepackage{amsmath, latexsym}
\usepackage{graphicx}
\usepackage{tikz}
\usepackage{listings}
\usepackage{verbatim}
\usepackage{bigints}
\usepackage{float}
\usepackage{titling}
\usepackage[percent]{overpic}
\usepackage[para]{footmisc}
\usepackage[colorinlistoftodos]{todonotes}
\usepackage[style=authoryear,sorting=nty,maxcitenames=1]{biblatex}
\addbibresource{references.bib}

\title{Kirchhoff formula for spherically symmetric acoustic problem}
\author{}
\date{\today}

\begin{document}
\maketitle
\section*{Formulation}
\begin{figure}[h!]\label{schematic}
    \centering
    \begin{tikzpicture}
        \draw [-stealth](0, 0) -- (8.0,0);
        \draw [black, thick, fill] (0, 0.0) circle (0.05);
        \draw [red, thick, fill] (3, 0.0) circle (0.05);
        \draw (4.5, 0.5) node {$\phi(r, t)$};
        \draw (5.5, -0.22) node {$r$};
        \draw (3.0, -0.22) node {$R$};
    \end{tikzpicture}
\end{figure}
In this work we derive the Kirchhoff integral formula for spherically symmetric acoustic problems. We assume the wave propogates linearly in the region $r > R$ and write down the spherically symmetric wave equation
\begin{equation}\label{wave}
    \frac{1}{c^2}(\phi)_{tt} - (\phi)_{rr} - \frac{2\phi_r}{r} = 0 \;\;\;\; r > R.
\end{equation}
We define a generalised variable $\phi H(r - R)$ using the Heaviside function:
\begin{equation}
	\phi H(r - R) =\begin{cases}
	\phi , & \text{for $r > R$}.\\
	0, & \text{for $r < R$ }.
	\end{cases}
\end{equation}
The wave equation in generalised variable is given by
\begin{equation}\label{generalised_wave}
    \frac{1}{c^2}(\phi H(r - R))_{tt} - (\phi H(r - R))_{rr} - \frac{2(\phi H(r - R))_r}{r} = s(r, t) \;\;\;\; r > 0.
\end{equation}
Where $s(r, t)$ is the source term given by
\begin{equation}
    s(r,t) = - \phi_{r}\delta(r - R) - (\phi \delta (r - R))_r - \frac{2\phi \delta(r - R)}{r}.
\end{equation}

Unlike (\ref{wave}), the generalised wave equation $(\ref{generalised_wave})$ is valid over the entire space. So we can use the free space Green's function to solve the equation. The Green's function for the wave operator in ($\ref{wave}$) is given by
\begin{equation}
    G(r, t; r', t') = \frac{1}{4\pi |r - r'|}\delta \Big(t - t' - \frac{|r - r'|}{c}\Big). 
\end{equation}
We can solve the wave equation (\ref{generalised_wave}) by convolution of source and the Green's function.
\begin{equation}
    \phi(r, t) H(r - R) = \int_{t'}\int_{r'}s(r', t')G(r, t; r', t')dr'dt'
\end{equation}
\begin{equation}
    \begin{split}
        \phi H(r - R) = &-\frac{1}{4\pi}\int_{t'}\int_{r'} \phi_{r'}\delta(r' - R) \frac{\delta \Big(t - t' - \frac{|r - r'|}{c}\Big)}{|r - r'|} dr'dt'\\
                        &-\frac{1}{4\pi}\int_{t'}\int_{r'} (\phi \delta (r' - R))_{r'} \frac{\delta \Big(t - t' - \frac{|r - r'|}{c}\Big)}{ |r - r'| }dr'dt'\\
                        &-\frac{1}{4\pi}\int_{t'}\int_{r'} \frac{2\phi \delta(r' - R)}{r'}\frac{\delta \Big(t - t' - \frac{|r - r'|}{c}\Big)}{ |r - r'| }dr'dt'.
    \end{split}
\end{equation}
We simplify the above equations using the delta function property
\begin{equation}
    \phi H(r - R) = -\frac{\phi_{r}(R, \tau)}{4\pi  |r - R|} - \frac{1}{4\pi}\Big(\frac{\phi (R, \tau)}{|r - R|}\Big)_r   - \frac{1}{4\pi} \frac{2\phi(R, \tau)}{R |r - R|}.
\end{equation}
Where $\tau = t - \frac{|r - R|}{c}$ is the retarded time. Simplifying further we obtain
\begin{equation}
    \phi(r, t) = -\frac{1}{4\pi (r - R)}\Big[ \phi_{r}(R, \tau) - \frac{1}{c}\phi_t (R, \tau) - \frac{\phi(R, \tau)}{(r - R)}  + \frac{2 \phi(R, \tau)}{R} \Big] \;\;\;\;\; r > R.
\end{equation}



\end{document}

